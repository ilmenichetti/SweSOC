\documentclass[]{tufte-handout}

% ams
\usepackage{amssymb,amsmath}

\usepackage{ifxetex,ifluatex}
\usepackage{fixltx2e} % provides \textsubscript
\ifnum 0\ifxetex 1\fi\ifluatex 1\fi=0 % if pdftex
  \usepackage[T1]{fontenc}
  \usepackage[utf8]{inputenc}
\else % if luatex or xelatex
  \makeatletter
  \@ifpackageloaded{fontspec}{}{\usepackage{fontspec}}
  \makeatother
  \defaultfontfeatures{Ligatures=TeX,Scale=MatchLowercase}
  \makeatletter
  \@ifpackageloaded{soul}{
     \renewcommand\allcapsspacing[1]{{\addfontfeature{LetterSpace=15}#1}}
     \renewcommand\smallcapsspacing[1]{{\addfontfeature{LetterSpace=10}#1}}
   }{}
  \makeatother

\fi

% graphix
\usepackage{graphicx}
\setkeys{Gin}{width=\linewidth,totalheight=\textheight,keepaspectratio}

% booktabs
\usepackage{booktabs}

% url
\usepackage{url}

% hyperref
\usepackage{hyperref}

% units.
\usepackage{units}


\setcounter{secnumdepth}{-1}

% citations


% pandoc syntax highlighting

% table with pandoc

% multiplecol
\usepackage{multicol}

% strikeout
\usepackage[normalem]{ulem}

% morefloats
\usepackage{morefloats}


% tightlist macro required by pandoc >= 1.14
\providecommand{\tightlist}{%
  \setlength{\itemsep}{0pt}\setlength{\parskip}{0pt}}

% title / author / date
\title{Model draft definition}
\author{Lorenzo Menichetti}
\date{1/10/2023}


\begin{document}

\maketitle




\hypertarget{model-definitions}{%
\section{Model definitions}\label{model-definitions}}

The model components are:

The decomposition model

The two porosity functions, which are variables in the decomposition
model

Accessory functions to calculate the parameters for the porosity
functions

\hypertarget{decomposition-model}{%
\subsection{Decomposition model}\label{decomposition-model}}

The decomposition model is a compartmental model considering four
components, two pools (Young and Old) for the mesopore fraction and two
for the micropore fraction

\begin{equation}
    \begin{cases}
      \frac{dM_{Y_{(mes)}}}{dt} = I_m + \left( \frac{\phi_{mes}}{\phi_{mes}+\phi_{mic}}\right) \cdot I_r - k_Y \cdot M_{Y_{(mes)}}+ T_Y \\
      
      \frac{dM_{O_{(mes)}}}{dt} = \left( \epsilon \cdot k_Y \cdot M_{Y_{(mes)}} \right) - \left( (1- \epsilon) \cdot k_O \cdot M_{O_{(mes)}} \right) + T_O\\
      
      \frac{dM_{Y_{(mic)}}}{dt} = \left( \frac{\phi_{mes}}{\phi_{mes}+\phi_{mic}}\right) \cdot I_r - k_Y \cdot F_{prot} \cdot M_{Y_{(mes)}}+ T_Y \\
      
      \frac{dM_{O_{(mic)}}}{dt} = \left( \epsilon \cdot k_Y \cdot F_{prot} \cdot M_{Y_{(mes)}} \right) - \left( (1- \epsilon) \cdot k_O \cdot F_{prot} \cdot M_{O_{(mes)}} \right) + T_O
    \end{cases}\
\end{equation}

The model has a feedback mechanism at the level of porosities
\(\phi_n\), so the system is instead:

\begin{equation}
    \begin{cases}
      \frac{dM_{Y_{(mes)}}}{dt} = I_m + \left( \frac{\phi_{mes}(M_{Y_{(mes)}}, M_{O_{(mes)}},M_{Y_{(mic)}}, M_{O_{(mic)}})}{\phi_{mes}(M_{Y_{(mes)}}, M_{O_{(mic)}},M_{Y_{(mic)}}, M_{O_{(mic)}})+\phi_{mic}(M_{Y_{(mic)}}, M_{O_{(mic)}})}\right) \cdot I_r - k_Y \cdot M_{Y_{(mes)}}+ T_Y \\
      
      \frac{dM_{O_{(mes)}}}{dt} = \left( \epsilon \cdot k_Y \cdot M_{Y_{(mes)}} \right) - \left( (1- \epsilon) \cdot k_O \cdot M_{O_{(mes)}} \right) + T_O\\
      
      \frac{dM_{Y_{(mic)}}}{dt} = \left( \frac{\phi_{mes}(M_{Y_{(mes)}}, M_{O_{(mes)}},M_{Y_{(mic)}}, M_{O_{(mic)}})}{\phi_{mes}(M_{Y_{(mes)}}, M_{O_{(mic)}},M_{Y_{(mic)}}, M_{O_{(mic)}}) +\phi_{mic}(M_{Y_{(mic)}}, M_{O_{(mic)}})}\right) \cdot I_r - k_Y \cdot F_{prot} \cdot M_{Y_{(mes)}}+ T_Y \\
      
      \frac{dM_{O_{(mic)}}}{dt} = \left( \epsilon \cdot k_Y \cdot F_{prot} \cdot M_{Y_{(mes)}} \right) - \left( (1- \epsilon) \cdot k_O \cdot F_{prot} \cdot M_{O_{(mes)}} \right) + T_O
    \end{cases}\
\end{equation}

Where it appears clear that the two porosity terms, \(\phi_{mes}\) and
\(\phi_{mic}\), are dependent on the variation of the different C pools,
introducing a nonlinearity in the system.

\hypertarget{climatic-scaling-of-decomposition}{%
\subsubsection{Climatic scaling of
decomposition}\label{climatic-scaling-of-decomposition}}

The climatic scaling of decomposition can be introduced in the model as
in the ICBM model from which this model is derived, with a multiplier
oscillating around 1 (which is the standard climate in Ultuna, where the
model was developed) that multiplies the two kinetics \(k_Y\) and
\(k_O\). The calculation of this term, called \(r_e\), is already
developed (\url{https://github.com/ilmenichetti/reclim}), but adding
this scaling can require a recalibration of the model on multiple sites.

\hypertarget{porosity-variation-functions}{%
\subsection{Porosity variation
functions}\label{porosity-variation-functions}}

We need to define the functions of variation of the two porosities
considered, mesopores \(\phi_{mes}\) and micropores \(\phi_{mic}\).
These depends on some external constants, and ultimately on the
variation of the organic C pools.

\hypertarget{microporosity-function}{%
\subsubsection{Microporosity function}\label{microporosity-function}}

The microporosity depends on the variation of the organic matter
associated with micropores, \(M_{Y_{(mic)}}\) and \(M_{O_{(mic)}}\),
plus some additional constants:

\[ \phi_{mic} = \frac{\left[ f_{agg} \left(  \frac{M_{Y_{mic}}+ M_{O_{mic}}}{ \gamma_o} \right)\right] + (F_{text_{mic}} \Delta_z \phi_{min})}{\Delta_z}\]
(eq. 24) Where \(\Delta_z\) is the depth considered, \(F_{text_{mic}}\)
is the proportion of the textural pore spaces that comprises micropores
and \(f_{agg}\) is an aggregation factor defined as the slope of the
linear relationship assumed between the volume of aggregation pore
spaces and the volume of organic matter. The constant \(F_{text_{mic}}\)
is calculated according to what specified in the section
\protect\hyperlink{textural-pore-space-function}{Textural pore space
function} while the constant \(f_{agg}\) is calculated according to what
specified in the section
\protect\hyperlink{porosity-slope-calculation}{Porosity slope
calculation}. \(\gamma_o\) is the density of organic matter (kg
m\(^{-3}\))

\hypertarget{mesopore-function}{%
\subsubsection{Mesopore function}\label{mesopore-function}}

The mesopore porosity function is a result of micropore and matrix
porosities: \[ \phi_{mes} = \phi_{mat}-\phi_{mic} \] (eq. 25) Matrix
porosity depends on all the organic pools plus additional constants:
\[\phi_{mat} = \frac{\left[ f_{agg} \frac{M_{s_{0}}}{ \gamma_o} \right] + \Delta_z \phi_{min})}{\Delta_z}\]
The variable \(M_{s_{0}}\) is the sum of all the organic matter pools:
\[M_{s_{0}} = M_{Y_{(mes)}} + M_{O_{(mic)}} + M_{Y_{(mic)}} + M_{O_{(mic)}}\]

\hypertarget{textural-pore-space-function}{%
\subsection{Textural pore space
function}\label{textural-pore-space-function}}

\[F_{text_{mic}} = \]

\hypertarget{porosity-slope-calculation}{%
\subsection{Porosity slope
calculation}\label{porosity-slope-calculation}}

\[f_{agg} = (y2-y1)/(x2-x1)\] Where \(x_n\) and \(y_n\) are the
coordinates of the volume of the aggregation pore space \(V_{agg}\) and
the volume of organic matter \(V_{s_o}\).

\hypertarget{volume-of-pore-space-and-organic-matter-calculation}{%
\subsection{Volume of pore space and organic matter
calculation}\label{volume-of-pore-space-and-organic-matter-calculation}}



\end{document}
